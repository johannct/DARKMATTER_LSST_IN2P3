\documentclass[12pt]{article}

\usepackage[sort&compress]{natbib}
\usepackage{graphicx}
\usepackage{authblk}

\usepackage[utf8]{inputenc}
\usepackage[T1]{fontenc}
\usepackage[french]{babel}
%\usepackage{french} 


\usepackage{macros}
\usepackage{aas_macros}


%% \usepackage[utf8]{inputenc}
%% \usepackage[T1]{fontenc}
%% \usepackage[french]{babel}
%% %\usepackage{french}

%% \usepackage{times}
\usepackage{geometry}
\geometry{a4paper, portrait, margin=2cm}

\usepackage{enumitem,amssymb}
\setenumerate{itemsep=0mm}
\usepackage{ragged2e}
%% \usepackage{graphicx}
%% \usepackage{comment}
\usepackage{multicol}
%\usepackage[usenames]{xcolor} %used for font color              
\definecolor{xlinkcolor}{cmyk}{1,1,0,0}
\usepackage{url}
\usepackage[
 colorlinks=true,    % false: boxed links; true: colored links 
 linkcolor=xlinkcolor,     % color of internal links            
 citecolor=xlinkcolor,     % color of links to bibliography
 filecolor=xlinkcolor,  % color of file links 
 urlcolor=xlinkcolor,      % color of external link
 final=true
]{hyperref}
%% \usepackage[super,sort&compress]{natbib}
%% \usepackage{enumitem}
%% \setenumerate{itemsep=0mm}

\newcommand{\Comment}[3]{\textcolor{#1}{(#2: #3)}} 
\newcommand{\ADW}[1]{\Comment{blue}{ADW}{#1}} % Alex Drlica-Wagner 
\newcommand{\KB}[1]{\Comment{orange}{KB}{#1}} % Keith Bechtol

%% \newlist{thematic}{itemize}{8}
%% \setlist[thematic]{label=$\square$}
%% \usepackage{pifont}
\newcommand{\cmark}{\ding{51}}%
\newcommand{\xmark}{\ding{55}}%
\newcommand{\done}{\rlap{$\square$}{\raisebox{2pt}{\large\hspace{1pt}\cmark}}%
\hspace{-2.5pt}}
\newcommand{\wontfix}{\rlap{$\square$}{\large\hspace{1pt}\xmark}}

\setlength{\parskip}{0.5em}

\begin{document}

\begin{raggedright} 
% part of template, but does not look good
  \large
  Exercice de prospective nationale en physique nucléaire, physique des particules et astroparticules\hfill ~ \linebreak
\end{raggedright}
\normalsize

\centering{\Large Identification de la Matière Noire à l'ère du grand relevé optique LSST}
\vspace{1cm}


%% \noindent \textbf{Thematic Areas:} \hspace*{58pt} $\square$ Planetary Systems \hspace*{12pt} $\square$ Star and Planet Formation  \hfill ~\linebreak
%% $\blacksquare$ Formation and Evolution of Compact Objects \hspace*{31pt} $\blacksquare$ Cosmology and Fundamental Physics \hfill ~ \linebreak
%%   $\square$  Stars and Stellar Evolution \hspace*{1pt} $\square$ Resolved Stellar Populations and their Environments \hspace*{40pt} \hfill ~\linebreak
%%   $\square$    Galaxy Evolution   \hspace*{45pt} $\square$             Multi-Messenger Astronomy and Astrophysics \hspace*{65pt} \hfill ~ \linebreak

%% Author list file generated with: mkauthlist.py UNKNOWN 
% python code/mkauthlist.py -f -s -j emulateapj -a data/order.csv data/astro2020_endorsers_trimmed_merged.csv tmp.tex 

\def\altaffilmark#1{\textsuperscript{#1}}
\def\affil#1{\noindent #1 \\}

%% \noindent \textbf{Principal Authors:} 
%% Keith Bechtol\altaffilmark{1}, \href{mailto:kbechtol@wisc.edu}{kbechtol@wisc.edu} (UW Madison)\\
%% Alex~Drlica-Wagner\altaffilmark{2,3,4}, \href{mailto:kadrlica@fnal.gov}{kadrlica@fnal.gov} (Fermila/KICP/UChicago)
%
%\noindent Name:	
% \linebreak						
%Institution:  
% \linebreak
%Email: 
% \linebreak
%Phone:  

\begin{flushleft}
  {\bf Auteur principal}  Johann Cohen-Tanugi\\
  {\bf Institution}  LUPM, université de Montpellier et CNRS/IN2P3, Montpellier, France\\
  {\bf email} johann.cohen-tanugi@umontpellier.fr
\end{flushleft}       


\noindent {\bf Co-auteurs :}
\begin{raggedright}
\small
Marc Moniez\altaffilmark{3},
%\setlength{\parskip}{\baselineskip}
\end{raggedright}

\noindent {\bf Endosseurs :}
\begin{raggedright}
\small
Céline Combet\altaffilmark{2},
David Maurin\altaffilmark{2},
%\setlength{\parskip}{\baselineskip}
\end{raggedright}

%\begin{multicols}{2}
\scriptsize
%\parskip=4pt

\affil{$^{1}$ LUPM, université de Montpellier et CNRS/IN2P3, Montpellier, France}
\affil{$^{2}$ LPSC, université de Grenoble et CNRS/IN2P3, Grenoble, France}
\affil{$^{3}$ LAL, université Paris-Orsay et CNRS/IN2P3, Orsay, France}

\normalsize
%\end{multicols}
%\parskip=8pt


%\noindent \textbf{Abstract:}
\abstract{Astrophysical observations currently provide the only robust, empirical measurements of dark matter. 
%Astrophysical probes will continue to guide other experimental efforts into the next decade, 
%In the coming decade, astrophysical probes of dark matter will explore parameter space beyond the current sensitivity of the high-energy physics program and will complement future experimental searches.
%Future observations with the Large Synoptic Survey Telescope (LSST) will provide necessary guidance for the experimental dark matter program. 
In the coming decade, astrophysical observations will guide other experimental efforts, while simultaneously probing unique regions of dark matter parameter space.
This white paper summarizes astrophysical observations that can constrain the fundamental physics of dark matter in the era of LSST. 
We describe how astrophysical observations will inform our understanding of the fundamental properties of dark matter, such as particle mass, self-interaction strength, non-gravitational interactions with the Standard Model, and compact object abundances. Additionally, we highlight theoretical work and experimental/observational facilities that will complement LSST to strengthen our understanding of the fundamental characteristics of dark matter.\footnote{Cette contribution s'inspire largement du {\it  Science Whitepaper for Astro 2020} \citep{arxiv.org/abs/1903.04425}, lui-même un résumé du livre blanc \citet[arxiv.org/abs/1902.01055], auxquels plusieurs des auteurs principaux de cette contribution ont participé.}\footnote{More information on the LSST dark matter effort can be found at \href{https://lsstdarkmatter.github.io/}{https://lsstdarkmatter.github.io/}}.
}

%\pagebreak

% Insert your white paper text here (max 5 pages inc. figs).

%\vspace{-1em}
\subsection*{Summary}

More than 85 years after its astrophysical discovery, the fundamental nature of dark matter remains one of the foremost open questions in science.
% physics and astronomy
Over the last several decades, an extensive experimental program has sought to determine the cosmological origin, constituents, and interaction mechanisms of dark matter. 
%While the existing experimental program has largely focused on weakly-interacting massive particles, there is strong theoretical motivation to explore a broader set of dark matter candidates.
%As the high-energy physics program expands to ``search for dark matter along every feasible avenue'' \citep{P5Report}, it is essential to keep in mind that the only direct, empirical measurements of dark matter properties to date come from astrophysical and cosmological observations.
%More than 85 years after its astrophysical discovery, the fundamental nature of dark matter remains one of the foremost open questions in physics and astronomy.
%Over the last several decades, an extensive experimental program has sought to determine the cosmological origin, fundamental constituents, and interaction mechanisms of dark matter.
To date, the only direct, positive empirical measurements of dark matter come from astrophysical observations.
%Discovering the fundamental nature of dark matter will draw upon the tools of particle physics, cosmology, stellar astrophysics, and galaxy evolution.
Discovering the fundamental nature of dark matter will necessarily draw upon the tools particle physics, cosmology, and astronomy.

LSST will provide a unique and impressive platform to study dark sector physics in the 2020s.
Originally envisioned as the ``Dark Matter Telescope'' \citep{Tyson:2001}, LSST will enable precision tests of the \LCDM model and elucidate the connection between luminous galaxies and the cosmic web of dark matter. 
Cosmology has consistently shown that it is impossible to separate the \emph{macroscopic distribution} of dark matter from the \emph{microscopic physics} governing dark matter.
In fact, some microscopic characteristics of dark matter are \emph{only accessible} via astrophysics.
Studies of dark matter, dark energy, massive neutrinos, and galaxy evolution are \emph{extremely complementary} from both a technical and scientific standpoint. 
%Originally envisioned as the ``Dark Matter Telescope'' \citep{Tyson:2001}, LSST will enable precision tests of the standard \LCDM model, 
%Originally envisioned as the ``Dark Matter Telescope'' \citep{Tyson:2001}, LSST will test a broad range of well-motivated theoretical models of dark matter including self-interacting dark matter, warm dark matter, dark matter-baryon scattering, ultra-light dark matter, axion-like particles, and primordial black holes. 
%In the precision era of LSST, studies of dark matter and dark energy are \emph{extremely complementary} from both a technical and scientific standpoint
%, a major joint experimental effort between NSF and DOE,
%The Large Synoptic Survey Telescope (LSST) provides a unique and impressive platform to study dark sector physics.
%LSST was originally envisioned as the ``Dark Matter Telescope'' \citep{Tyson:2001}, though in recent years, studies of fundamental physics with LSST have been more focused on dark energy.
%The Large Synoptic Survey Telescope (LSST), a major joint experimental effort between NSF and DOE, provides a unique and impressive platform to study dark sector physics.
%LSST was originally envisioned as the ``Dark Matter Telescope'' \citep{Tyson:2001}, though in recent years, studies of fundamental physics with LSST have been more focused on dark energy.
%Dark matter is an essential component of the standard \LCDM model, and a detailed understanding of dark energy cannot be achieved without a detailed understanding of dark matter.
%In the precision era of LSST, studies of dark matter and dark energy are \emph{extremely complementary} from both a technical and scientific standpoint.
%In addition, cosmology has consistently shown that it is impossible to separate the \emph{macroscopic distribution} of dark matter from the \emph{microscopic physics} governing dark matter.
A robust dark matter program leveraging LSST data has the ability to test a broad range of well-motivated theoretical models of dark matter including self-interacting dark matter, warm dark matter, dark matter-baryon scattering, ultra-light dark matter, axion-like particles, and primordial black holes. 

LSST will enable studies of Milky Way satellite galaxies, stellar streams, and strong lens systems to detect and characterize the smallest dark matter halos, thereby probing the minimum mass of ultra-light dark matter and thermal warm dark matter.
Precise measurements of the density and shapes of dark matter halos in dwarf galaxies and galaxy clusters will be sensitive to dark matter self-interactions probing hidden sector and dark photon models.
Microlensing measurements will directly probe primordial black holes and the compact object fraction of dark matter at the sub-percent level over a wide range of masses.
Precise measurements of stellar populations will be sensitive to anomalous energy loss mechanisms and will constrain the coupling of axion-like particles to photons and electrons.
Measurements of large-scale structure will spatially resolve the influence of both dark matter and dark energy, enabling searches for correlations between the two known components of the dark sector.
In addition, complementarity between astrophysical, direct detection, and other indirect searches for dark matter will help constrain dark matter-baryon scattering, dark matter self-annihilation, and dark matter decay. 

%Studies of dark matter with LSST will provide critical information about the fundamental nature of dark matter over the next decade at a low cost by leveraging a soon-to-exist facility.
%By leveraging a soon to-exist-facility, a small program with LSST will provide critical information about the fundamental nature of dark matter over the next decade at a low cost. 
%The study of dark matter with LSST presents a small experimental program that is guaranteed to provide critical information about the fundamental nature of dark matter over the next decade.
%LSST will rapidly produce high-impact science on the nature of fundamental dark matter by exploiting a soon-to-exist facility. 
Astrophysical dark matter studies will explore parameter space beyond the current sensitivity of the high-energy physics program and will complement other experimental searches.
This has been recognized in Astro 2010 \citep{Astro2010}, during the Snowmass Cosmic Frontier planning process \citep[][]{1310.8642, 1310.5662, 1305.1605}, in the P5 Report \citep[]{P5Report}, and in a series of recent Cosmic Visions reports \citep[][]{1604.07626,1802.07216}, including the ``New Ideas in Dark Matter 2017:\ Community Report'' \citep{Battaglieri:2017aum}.
%Astrophysical probes provide the only constraints on the minimum and maximum mass scale of dark matter, and 
%Astrophysical observations will likely continue to guide other experimental efforts.
%the experimental particle physics program for years to come.
In the 2020s, the impact of the LSST dark matter program will be enhanced by access to wide-field massively multiplexed spectroscopy on medium- to large-aperture telescopes ($\roughly 8$--$10$-meter class), deep spectroscopy on giant segmented mirror telescopes ($\roughly 30$-m class), together with high-resolution optical and radio imaging.
%with relatively smaller fields of view
Further theoretical work is also needed to interpret those observations in terms of particle models, to combine results from multiple observational methods, and to develop novel probes of dark matter.

This whitepaper is a summary of Drlica-Wagner et al. (2019) \citep{drlica-wagner_2019_lsst_dark_matter}.

\begin{table}[ht]
\footnotesize
\begin{center}
\begin{tabular}{l c c c}
\hline 
Model & Probe & Parameter & Value \\
\hline 
\hline
Warm Dark Matter  & Halo Mass & Particle Mass & $m \sim 18 \keV$ \\
Self-Interacting Dark Matter & Halo Profile & Cross Section & $\sigmam \sim 0.1\text{--}10\cm^2/\g$ \\
Baryon-Scattering Dark Matter & Halo Mass & Cross Section & $\sigma \sim 10^{-30} \cm^2$ \\
Axion-Like Particles & Energy Loss & Coupling Strength & $g_{\phi e} \sim 10^{-13} $ \\
Fuzzy Dark Matter & Halo Mass & Particle Mass & $m \sim 10^{-20} \eV$  \\
Primordial Black Holes  & Compact Objects & Object Mass & $M > 10^{-4} \Msun$ \\
WIMPs & Indirect Detection & Cross Section & $\sigmav \sim 10^{-27} \cm^3/\second$ \\
Light Relics & Large-Scale Structure & Relativistic Species & $N_{\rm eff} \sim 0.1$ \\[+0.5em]
\hline
\end{tabular}
\end{center}
\vspace{-1em}
\caption{\label{tab:models} Probes of fundamental dark matter physics in the LSST era, organized by dark matter model and assotiated observables. Sensitivity forecasts appear in the rightmost column.}
%Probes of fundamental dark matter physics with LSST. The four columns indicate classes of dark matter models, primary observational probe, corresponding dark matter parameters, and the estimated senstivity of LSST.}
%Sensitivity forecasts of Probes of fundamental dark matter physics in the LSST era. 


%Classes of dark matter models are listed in Column 1, and the primary observational probe that is sensitive to each model is listed in Column 2. The corresponding dark matter parameters are listed in Column 3, and estimates of LSST's senstivity to each parameter are listed in Column 4.}
\end{table}

\vspace{-1em} \subsection*{Modèles de Matière Noire} \vspace{-0.5em}

Si la décennie passée a cherché avant tout à détecter les signatures attendues d'un candidat de type WIMP ({\it weakly interactive massive particle}), la décennie qui s'ouvre devant nous fait face, du fait de l'échec de ces tentatives de détection, à une pléthore de modèles phénoménologiques proposant une réponse à la question de la nature de la matière noire. Si il est essentiel de concevoir l'instrumentation dédiée à la détection du plus grand nombre possible de ces candidats, il est important de rappeler que les observations astrophysiques sondent la physique de la matière noire au travers de son impact sur la formation des structures tout au long de l'histoire de l'Univers. Aux larges échelles, les données d'observations sont très bien décrites par l'ajustement statistique d'un modèle cosmologique qui inclue une composante non relativiste, non collisionnelle, et froide (CDM pour {\it cold dark matter}). Toutefois, de nombreuses extensions au modèle standard fournissent des candidats matière noire viables, qui dévieraient du scénario CDM standard. En effet les propriétés fondamentales d'une particule de matière noire -masse, sections efficaces d'interactions, couplage aux particules du modèle standard, évolution temporelle- peuvent laisser une trace détectable sur la distribution macroscopique de matière noire. Soutenu par la poursuite des efforts théoriques et l'appui de suivis observationnels sur d'autres télescopes, LSST sera sensible à plusieurs classes distinctes de modèles, par exemple :

\noindent \textbf{Matière Noire de type particule} LSST, combinée avec d'autres observations, pourra apporter des indices importants pour caractériser le section efficace d'auto-interaction et/ou de diffusion avec des baryons, la masse, le taux d'annihilation ou de désintégration, etc... De telles mesures viendront compléter et guider les efforts de détection directe, indirecte, et sur collisionneur.

\noindent \textbf{Matière Noire de type ondes} Les ALP ({\it Axion-Like Particles}) et autres candidats dits ``ultra-légers'' sont une alternative crédible aux candidats plus conventionnels. LSST sera en mesure d'apporter des contraintes distinctes sur la masse minimale de tels candidats, ainsi que sur le couplage ALP-particules du modèle standard.

\noindent \textbf{Objets compacts:} Cette classe de candidats est fondamentalement différente des deux premières. En particulier les trous noirs primordiaux (PBH pour {\it Primordial Black Hole}) ne sont accessibles à l'étude que par des observations astrophysiques, et des contraintes portées sur leur abondance fourniraient directement des contraintes sur l'amplitude des fluctuations de densité initiales, en plus de fournir des indices uniques sur la physique à ultra-haute énergie.

%Fundamental properties of dark matter---e.g., particle mass, self-interaction cross section, coupling to the Standard Model, and time evolution---can imprint themselves on the macroscopic distribution of dark matter in a detectable manner. With supporting theoretical efforts and follow-up observations, LSST will be sensitive to several distinct classes of dark matter models, including particle dark matter, field dark matter, and compact objects (\tabref{models}).

\vspace{-1em} \subsection*{Sondes de Matière Noire} \vspace{-0.5em}

\noindent {\bf Minimum Halo Mass:}
The standard cosmological model predicts a nearly scale-invariant mass spectrum of dark matter halos down to Earth-mass scales (or below), e.g., in WIMP and non-thermal axion models \citep{Green:2003un,2005Natur.433..389D,1412.5930}.
%A defining prediction of the standard cosmological model with cold dark matter (CDM) is the gathering of dark matter into gravitationally bound halos having a nearly scale-invariant mass spectrum on physical scales ranging from galaxy clusters to planet-scale masses.
%The cold, collisionless model of dark matter predicts that dark matter halos should exist down to Earth-mass scales (or below) in WIMP and non-thermal axion models \citep{Green:2003un,2005Natur.433..389D,1412.5930}.
Modifications to the cold, collisionless dark matter paradigm can suppress the formation of dark matter halos on these small scales.
Current observations provide a robust measurement of the dark matter halo mass spectrum for halos with mass $> 10^{10}\Msun$, and the smallest known galaxies provide an existence proof for halos of mass $\roughly 10^8 \Msun - 10^9 \Msun$ \citep{2017MNRAS.467.2019R,behroozi2018,Jethwa:2018,Kim:2017iwr,Nadler:2018,1807.07093}. 
LSST will expand the census of ultra-faint satellite galaxies orbiting the Milky Way and enable statistical searches for extremely low-luminosity and low-surface brightness galaxies throughout the Local Volume.
By measuring the galaxy luminosity function at the extreme low-mass threshold of galaxy formation, LSST will test the abundance of dark matter halos at $\sim10^8 \Msun$.

LSST will probe dark matter halos below the threshold of galaxy formation with stellar streams and strongly lensed systems.
%LSST will enable searches for completely dark halos using purely gravitational observational signatures, e.g., stellar stream gaps and strong gravitational lensing anomalies.
%subhalos purely through their gravitational signatures
Galactic dark matter subhalos with masses as small as $10^5$--$10^6 \Msun$ passing a stellar stream are capable of producing detectable gaps in the stellar density \citep[][]{erkal2016,bovy:2017}.
%Deep and precise LSST photometry is expected to increase the contrast between streams and the contaminating Milky Way field stars, dramatically increasing our ability to detect density variations and thus leading to the identification of less prominent gaps created by low-mass perturbers
By identifying additional stellar streams and increasing the density contrast of known streams against the smooth Milky Way halo, LSST will shift analysis from individual gaps into the regime of subhalo population statistics and (in)consistency with cold dark matter predictions.
Importantly, LSST will allow studies of streams farther from the center of the Galaxy for which confounding baryonic effects are lessened.
%LSST will mitigate both of these issues by examining streams farther from the center of the Galaxy where these effects are lessened.
Meanwhile, strong gravitational lensing can be used to measure the abundance and masses of subhalos in massive galaxies and small isolated halos along the line of sight at cosmological distances, independent of their baryon content.
LSST will increase the number of lensed systems from the current sample of hundreds to an expected sample of thousands of lensed quasars \citep{O+M10} and tens of thousands of lensed galaxies \citep{Collett2015}.
%Through analysis of flux ratio anomalies, gravitational imaging, and measuring the power spectrum of 

\noindent {\bf Halo Profiles:}
Measurements of the radial density profiles and shapes of dark matter halos are sensitive to the microphysics governing non-gravitational dark matter self-interactions, which could produce flat density cores \citep{Spergel:1999mh} and more spherical halo shapes \citep{Peter:2013}.
%Dwarf galaxies, galaxy clusters, merging clusters.
Through galaxy-galaxy weak lensing, LSST will be able to distinguish cored versus cuspy NFW density profiles for a sample of low-redshift dwarf galaxies with masses $M_\text{halo} = 3\times10^9\,h^{-1}\Msun$.
Studies of the density profiles of massive galaxy clusters, as well as systems of merging galaxy clusters, will constrain the scattering cross section at the level $\sigmam \sim 0.1-1 \cmg$.
Measuring halo profiles over a range of mass scales will provide sensitivity to dark matter scattering with non-trivial velocity dependence.
%Due to the possibility that dark matter scattering has a non-trivial velocity dependence, it is important to probe halo profiles over a wide range of mass scales.
%The standard CDM model predicts that dark matter halos should be “cuspy, i.e. with inner densityprofiles asymptoting to high central densities. If dark matter is able to interact through scattering or the exchange of some light mediator, then the density of halos could instead flatten out to produce dark matter “cores”.

\noindent {\bf Compact Objects:} 
LSST has the ability to directly detect signals of compact halo objects through precise, short- ($\roughly 30 \second$) and long-duration ($\roughly1 \unit{yr}$) observations of classical and parallactic microlensing\citep{1509.04899}.
If scheduled optimally, LSST could extend PBH sensitivity to $\roughly0.03\%$ of the dark matter fraction for masses $\gtrsim 10^{-1} \Msun$.
By supplementing the LSST survey with astrometric microlensing observations, it will be possible to break lensing mass-geometry degeneracies and make precise measurements of individual black hole masses. Thus, if PBHs make up a significant fraction of dark matter, LSST will effectively measure their ``particle'' properties and provide insight into the fundamental physics of the early universe.


\noindent {\bf Anomalous Energy Loss:}
Observations of stars provide a mechanism to probe temperatures, particle densities, and time scales that are inaccessible to laboratory experiments. Since conventional astrophysics allows us to quantitatively model the evolution of stars, detailed study of stellar populations can provide a powerful technique to probe new physics. In particular, if new light particles exist and are coupled to Standard Model fields, their emission would provide an additional channel for stellar energy loss. 
LSST will greatly improve our understanding of stellar evolution by providing unprecedented photometry, astrometry, and temporal sampling for a large sample of faint stars. In particular, measurements of the white dwarf luminosity function, giant branch stars, and core-collapse supernovae will provide sensitivity to the axion-electron coupling.

\noindent {\bf Large-Scale Structure:} LSST will produce the largest and most detailed map of the distribution of matter and the growth of cosmic structure over the past 10 Gyr.
The large-scale clustering of matter and luminous tracers in the late-time universe is sensitive to the total amount of dark matter, the fraction of dark matter in light relics that behave as radiation at early times, and fundamental couplings between dark matter and dark energy.
Measurements of large-scale structure with LSST will enhance constraints on massive neutrinos and other light relics from the early universe that could compose a fraction of the dark matter.
Additionally, LSST will use supernovae and $3\times2$pt statistics of galaxy clustering and weak lensing to measure dark energy in independent patches across the sky, allowing for spatial cross correlation between dark matter and dark energy \citep{0902.2590}.

\begin{figure}[t]
\centering
\includegraphics[width=0.53\columnwidth]{figures/SIDM_WDM_figw_coll.pdf}
\includegraphics[width=0.46\columnwidth]{figures/WDM_SIDM_discovery_test.pdf}
\caption{\emph{Left}: Projected joint sensitivity to WDM particle mass and SIDM cross section from LSST observations of dark matter substructure. 
\emph{Right}: Example of a measurement of particle properties for a dark matter model with a self-interaction cross section and matter power spectrum cut-off just beyond current constraints ($\sigmam = 2 \cmg$ and $\mWDM = 6\keV$, indicated by the red star) \citep{drlica-wagner_2019_lsst_dark_matter}. Complementary observations can break degeneracies among dark matter models that have the same approximate behavior on small scales but differ in detail.}
\end{figure}

%\begin{figure}[t]
%\centering
%\includegraphics[width=0.49\textwidth]{figures/LSST_Mmin.pdf}
%\includegraphics[width=0.49\textwidth]{figures/streamgap_constraints_3.png}
%\includegraphics[width=0.50\textwidth]{figures/wdm_constraints_yh.png}
%\caption{Three complementary probes of minimum dark matter halo mass.}
%\end{figure}

%\begin{figure}
%    \centering
%    \includegraphics[width=0.50\textwidth]{figures/wdm_constraints_yh.png}
%    \caption{ \label{fig:lensing_wdmlim_vs_nlens} Projected $2\sigma$ constraints on WDM particle mass as a function of the number of strong lens systems that achieve a given (sub)halo mass detection threshold, under the assumption that CDM is correct. These constraints include only the contribution from halo substructure, and do not include the line-of-sight contribution.
%Exisiting Lyman-$\alpha$ forest constraints are shown with a dashed horizontal line \citep{2017PhRvD..96b3522I}. Figure based on \citet{Hezaveh_2016ltk}.
%}
%\end{figure}

%\begin{figure}[t]
%\centering
%\includegraphics[width=0.775\textwidth]{figures/LSST_Mmin.pdf}
%\caption{Forecast for the minimum dark matter subhalo mass probed by LSST via observations of Milky Way satellites. The red band shows the $95\%$ confidence interval from our MCMC fits to mock satellite populations as a function of the true peak subhalo mass necessary for galaxy formation. Note that we marginalize over the relevant nuisance parameters associated with the galaxy--halo connection---including the effects of baryons using a model calibrated on subhalo disruption in hydrodynamic simulations \citep{2018ApJ...859..129N}---in our sampling. We indicate the corresponding constraints on the warm dark matter mass assuming $M_{\rm hm} = \mathcal{M}_{\rm{min}}$ (see \secref{wdm})}\label{fig:satellite_mmin}
%\end{figure}

%\begin{figure}[t]
%\centering
%\includegraphics[width=0.85\textwidth]{figures/streamgap_constraints_3.png}
%\caption{\label{fig:streamsurveys} Detection limits for gaps formed from subhalos of different masses using photometry from SDSS (blue) or the 10-year LSST stack (green) as a function of the stream surface brightness.
%Shaded regions correspond to a 10-40 kpc distance range, with the lines representing 20 kpc. For streams with surface brightnesses similar to those found in DES, 32--$33 \magn \asec^{-2}$, LSST is expected to probe halo masses two to three orders of magnitude smaller than SDSS and substantially improve the current constraints from Milky Way satellites \citep{Nadler:2018, Jethwa:2018,Kim:2017iwr} and the Lyman-$\alpha$ forest \citep{2017PhRvD..96b3522I}. 
%We connect the detected halos to the mass of the warm dark matter particle that would produce a minimum halo of that mass using the relationship determined by \cite{Bullock:2017}. Note that the halo mass definition used here is the $z=0$ virial mass; to relate this quantity to the peak subhalo mass used in our warm dark matter constraints, we have assumed the best-case scenario of no tidal mass loss.
%}
%\end{figure}
\vspace{-1em} \subsection*{Complémentarité} \vspace{-0.5em}

Si LSST permet de mettre en oeuvre un programme de recherche très riche, il bénéficiera grandement de l'apport, souvent crucial, d'autres instruments et techniques. Il est à noter que ces derniers sont en général tout autant nécessaires pour enrichir le programme de recherche sur l'Energie Noire, au coeur de l'implication de l'IN2P3 dans la science avec LSST.
Par exemple, la spectroscopie grand-champ massivement multiplexée sur des télescopes de classe 8-10 mètres permet d'améliorer la caractérisation de la limite à petite masse des halos (masse minimale et profil).
L'étude du lentillage fort et micro, ainsi que des amas de galaxie, bénéficierait quant à elle grandement d'observations au sol (optique adaptative) ou spatiales à très haute résolution. Enfin, les recherches indirectes d'un signal de Matière Noire avec des télescopes neutrino ou gamma \citep{Charles:2016,Albert:2017,1404.5503} pourront s'appuyer sur le gain en précision de la cartographie par LSST de la distribution de masse aux échelles galactiques et extra-galactiques. LSST sera également capables de suivre des événements extrêmes du ciel transitoires, parmi lesquels peut se cacher une signature de candidats Matière Noire comme les ALP ({\it axion-like particles}) \citep{2017PhRvL.118a1103M}.
Il est important de rappeler que cette recherche d'un signal indirect est un des thèmes scientifiques historiques de l'axe ``astroparticules'' de l'IN2P3. Il en est de même des efforts de détection directe, pour lesquels LSST étendra les mesures de cinématique locales obtenues par Gaia à de bien plus grandes distances, et sondera avec les observations de structures à petite échelle un régime de masse et de section efficace des candidats particules de Matière Noire inaccessibles aux mesures de détection directe.

%\vspace{-1em} \subsection*{Outlook: Discovery Potential} \vspace{-0.5em}

\begin{figure}[t]
\centering
\includegraphics[width=0.49\textwidth]{figures/macho_limits.pdf}
\includegraphics[width=0.49\columnwidth]{figures/bsdm_limits.pdf}
\caption{\label{fig:macho_constraints}
    \emph{Left}: Constraints on the maximal fraction of dark matter in compact objects from existing probes (blue and gray) and projected sensitivity for LSST microlensing measurements (gold).
    %Existing constraints include: lack of extragalactic gamma-rays from PBH evaporation \citep[EGR;][]{0912.5297, 1604.05349}, gamma-ray femtolensing \citep[GF;][]{1204.2056}, neutron star capture \citep[NS][]{1301.4984}, M31 microlensing \citep[M31ML][]{1701.02151}, Milky Way microlensing \citep[MWML;][]{2007A&A...469..387T, 2001ApJ...550L.169A, 2009MNRAS.397.1228W}, lensing of supernovae \citep[LSN;][]{1712.02240,1712.06574}, Eridanus II and other dwarf-galaxy constraints \citep[EII;][]{2016ApJ...824L..31B, 1611.05052}, wide binary stars \citep[WB;][]{2009MNRAS.396L..11Q, 2004ApJ...601..311Y}, cosmic microwave background \citep[CMB;][]{2017PhRvD..95d3534A, 2008ApJ...680..829R}, and disk stability \citep[DS;][]{1985ApJ...299..633L, 1994ApJ...437..184X}.
    %To improve figure clarity we have not shown some astrophysical constraints where they are less sensitive than a presented constraint; see \citet{2016PhRvD..94h3504C} for a more complete review.
    %There are a range of constraints for most astrophysical probes in the literature due to varying assumptions within a single work (EGR, NS, and EII) and reanalysis/disagreements between groups (WB, CMB).
    %We present the most conservative constraints in blue and the most aggressive constraints in gray.
    %The LSST M31 microlensing projection is based on extrapolating HSC constraints \citep{1701.02151} assuming a 10-day mini-survey of M31 with a $12\second$ cadence between exposures.
    %% Such a survey is approximately 10 times longer with an order of magnitude faster cadence than the existing HSC survey.
    %The projected LSST Milky Way (MW) microlensing and paralensing constraints are from a Monte Carlo analysis where lenses were injected into light curves based on LSST OpSim cadence simulations
    %% (see \url{https://github.com/lsstdarkmatter/dark-matter-paper/issues/8} for details).
    %The paralensing constraint comes from assuming that only the secondary microlensing parallax signal is used for discovery, and not the primary heliocentric microlensing signal.
    \emph{Right}: Constraints on dark matter-baryon scattering through a velocity-independent, spin-independent contact interaction with protons from existing constraints (blue and gray) and projections for LSST observations of Milky Way satellite galaxies (gold) \citep{drlica-wagner_2019_lsst_dark_matter,Gluscevic:prep}.
    %Existing constraints are shown in blue and gray.
    %Existing constraints (shown in blue) include measurements of the CMB power spectrum \citep[CMB;][]{Gluscevic:2017ywp} and constraints from the X-ray Quantum Calorimeter experiment \citep[XQC;][]{0704.0794}. Direct detection constraints include results from CRESST-III \citep{1711.07692}, the CRESST 2017 surface run \citep{1707.06749}, and XENON1T \citep{1705.06655}, as interpreted by \citet[][]{1802.04764}. %\citep{2018PhRvD..97l3013K}.
    %Additional constraints that include the effects of cosmic-ray heating of dark matter are shown in gray \citep[][]{1810.10543}.
    %The projected sensitivity of LSST to dark matter-baryon scattering through observations of Milky Way satellite dwarf galaxies is shown in gold.
}
\end{figure}

\vspace{-1em} \subsection*{Conclusions} \vspace{-0.5em}

LSST commencera les opérations de science vers 2022. La France est très largement impliquée dans le projet, au niveau de la construction de la caméra et de la charge de calcul pour la réduction et l'analyse des données. Cette implication a historiquement trouvé son ancrage scientifique au sein de la communauté française intéressée par la question de l'énergie noire et membre de la {\it Dark Energy Science Collaboration} (DESC). DESC a récemment intégré un groupe de travail dédié à la Matière Noire, reconnaissant que cette thématique est indissolublement liée à celle de l'énergie noire\footnote{Le site de LSST considère le secteur sombre comme un seul axe thématique auquel LSST a été conçu pour répondre, voir \href{https://www.lsst.org/science}{https://www.lsst.org/science}. } et que les sondes ciblées et les outils d'analyses sont en grande partie semblables.
De son côté, l'IN2P3 a une riche contribution à la question essentielle de la nature de la matière noire (recherches directe, indirecte, sur collisionneur, et phénoménologie associée), et est idéalement positionné pour jouer un rôle important dans cette thématique avec LSST. Dans ce contexte, il est important de souligner la nécessité de construire communauté Matière Noire au sein de l'IN2P3 et du CNRS plus généralament, qui déborde le cadre des différents projets. 

De plus, il est à parier que de nouvelles idées verront le jour à l'ère de LSST, pour proposer de nouvelles sondes, où de nouvelles manières d'associer les différents efforts expérimentaux. Il est essentiel de maintenir une communauté vibrante et soudée afin de....Gageons que les observations astrophysiques futures joueront alors un rôle central.


\def\bibname{References}
\begingroup
  \small
  \setlength{\bibsep}{0pt plus 0.5ex}
  \bibliographystyle{JHEP}
  %\bibliographystyle{yahapj}
  \bibliography{main}
\endgroup

%\clearpage
%% \subsection*{Affiliations}
%% \input{institutionAliases}


\begin{multicols}{2}
\scriptsize
\parskip=4pt

\affil{$^{1}$ \UWMadison}
\affil{$^{2}$ \FNAL}
\affil{$^{3}$ \KICP}
\affil{$^{4}$ \UChicago}
\affil{$^{5}$ \UCI}
\affil{$^{6}$ \damtp}
\affil{$^{7}$ \KIPAC}
\affil{$^{8}$ \NYU}
\affil{$^{9}$ \Durham}
\affil{$^{10}$ \LLNL}
\affil{$^{11}$ \KASSI}
\affil{$^{12}$ \SISSA}
\affil{$^{13}$ \IFPU}
\affil{$^{14}$ \INFN}
\affil{$^{15}$ \SLAC}
\affil{$^{16}$ \GRAPPA}
\affil{$^{17}$ \Leiden}
\affil{$^{18}$ \JHU}
\affil{$^{19}$ \Port}
\affil{$^{20}$ \UCR}
\affil{$^{21}$ \UCLA}
\affil{$^{22}$ \OskarKlein}
\affil{$^{23}$ \INAFOATs}
\affil{$^{24}$ \EPFL}
\affil{$^{25}$ \CfA}
\affil{$^{26}$ \LBL}
\affil{$^{27}$ \daa}
\affil{$^{28}$ \UAS}
\affil{$^{29}$ \Rutgers}
\affil{$^{30}$ \Stanford}
\affil{$^{31}$ \PI}
\affil{$^{32}$ \CNRSA}
\affil{$^{33}$ NASA Goddard Space Flight Center}
\affil{$^{34}$ \UCBP}
\affil{$^{35}$ \OU}
\affil{$^{36}$ \LUPM}
\affil{$^{37}$ \MPE}
\affil{$^{38}$ \CMUCosmo}
\affil{$^{39}$ Institute for Theoretical Particle Physics and Cosmology, RWTH Aachen University, Germany}
\affil{$^{40}$ Univ. Grenoble Alpes, USMB, CNRS, LAPTh, F-74940 Annecy, France}
\affil{$^{41}$ \HarvardPhys}
\affil{$^{42}$ \UNM}
\affil{$^{43}$ \Queensland}
\affil{$^{44}$ \IFUNAM'}
\affil{$^{45}$ \UoM}
\affil{$^{46}$ \JPL}
\affil{$^{47}$ Laboratory for Astroparticle Physics, University of Nova Gorica}
\affil{$^{48}$ Department of Physics, University of Surrey, UK}
\affil{$^{49}$ \UCD}
\affil{$^{50}$ \Oxford}
\affil{$^{51}$ \CITA}
\affil{$^{52}$ \kavli}
\affil{$^{53}$ \IFT}
\affil{$^{54}$ \ANLHEP}
\affil{$^{55}$ Physical Science Department, Barry University}
\affil{$^{56}$ \UFL}
\affil{$^{57}$ \UR}
\affil{$^{58}$ \UGTO}
\affil{$^{59}$ \Haverford}
\affil{$^{60}$ \CCA}
\affil{$^{61}$ \OSU}
\affil{$^{62}$ \dunlap}
\affil{$^{63}$ \VT}
\affil{$^{64}$ \UPenn}
\affil{$^{65}$ Yonsei University, Seoul, South Korea}
\affil{$^{66}$ \UCSC}
\affil{$^{67}$ \TIFR}
\affil{$^{68}$ \ioa}
\affil{$^{69}$ \Brown}
\affil{$^{70}$ \BenGurion}
\affil{$^{71}$ \UCL}
\affil{$^{72}$ \UCB}
\affil{$^{73}$ \NOAO}
\affil{$^{74}$ \UNIPD}
\affil{$^{75}$ \UCSD}
\affil{$^{76}$ \Princeton}
\affil{$^{77}$ \CNYang}
\affil{$^{78}$ \Pitt}
\affil{$^{79}$ \VSI}
\affil{$^{80}$ \ICTP}
\affil{$^{81}$ Laboratoire de l'Accélérateur Linéaire, IN2P3-CNRS, France}
\affil{$^{82}$ \IUCAA}
\affil{$^{83}$ \Siena}
\affil{$^{84}$ \Wyoming}
\affil{$^{85}$ \Caltech}
\affil{$^{86}$ \Yale}
\affil{$^{87}$ \BNL}
\affil{$^{88}$ \Tamu}
\affil{$^{89}$ \ETH}
\affil{$^{90}$ Center for Cosmology and AstroParticle Physics, The Ohio State University}
\affil{$^{91}$ Department of Astronomy, The Ohio State University}
\affil{$^{92}$ \RomaS}
\affil{$^{93}$ \INFNRM}
\affil{$^{94}$ \UNH}
\affil{$^{95}$ \STSCI}
\affil{$^{96}$ Laborat\'orio Interinstitucional de e-Astronomia - LIneA, Rua Gal. Jos\'e Cristino 77, Rio de Janeiro, RJ - 20921-400, Brazil}
\affil{$^{97}$ \Sejong}
\affil{$^{98}$ \KSU}
\affil{$^{99}$ Departamento de F\'isica Te\'orica, M-15, Universidad Aut\'onoma de Madrid, E-28049 Madrid, Spain}
\affil{$^{100}$ \StonyBrook}
\affil{$^{101}$ \SHAO}
\affil{$^{102}$ \Carnegie}
\affil{$^{103}$ \UMich}
\affil{$^{104}$ \MIT}
\affil{$^{105}$ INAF-Italian National Institute of Astrophysics, Italy}
\affil{$^{106}$ Space Telescope Science Institute}
\affil{$^{107}$ \DukePhys}
\affil{$^{108}$ \Duke}
\affil{$^{109}$ \UGTO'}
\affil{$^{110}$ \houston}
\affil{$^{111}$ \Syracuse}
\affil{$^{112}$ \MPIA}
\affil{$^{113}$ Center for Astrophysics and Cosmology, University of Nova Gorica}
\affil{$^{114}$ \ED}

\normalsize
\end{multicols}
\parskip=8pt




\end{document}

